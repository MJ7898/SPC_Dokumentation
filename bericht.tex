%%%%%%%%%%%%%%%%%%%%%%%%%%%%%%%%%%%%%%%%%%%%%%%%%%%%%%%%%%%%%%%%%%%%%%%%%%%%%%%
%% Descr:       Vorlage für Berichte der DHBW-Karlsruhe
%% Author:      Prof. Dr. Jürgen Vollmer, juergen.vollmer@dhbw-karlsruhe.de
%% $Id: bericht.tex,v 1.24 2018/10/23 09:10:06 vollmer Exp $
%%  -*- coding: utf-8 -*-
%%%%%%%%%%%%%%%%%%%%%%%%%%%%%%%%%%%%%%%%%%%%%%%%%%%%%%%%%%%%%%%%%%%%%%%%%%%%%%%

\documentclass[
   ngerman          % neue deutsche Rechtschreibung
  ,a4paper          % Papiergrösse
% ,twoside          % Zweiseitiger Druck (rechts/links)
% ,10pt             % Schriftgrösse
  ,11pt
% ,12pt
  ,pdftex
%  ,disable         % Todo-Markierungen auschalten
]{report}

% Bitte die Codierung Ihrer Dateien auswählen:
% \usepackage[latin1]{inputenc}    % Für UNIX mit ISO-LATIN-codierten Dateien
% \usepackage[applemac]{inputenc}  % Für Apple Mac
% \usepackage[ansinew]{inputenc}   % Für Microsoft Windows
\usepackage[utf8]{inputenc}        % UTF-8 codierte Dateien
                                   % Dieses Dokument ist unter Unix erstellt, daher
                                   % wird diese Input-Codierung benutzt.

\usepackage{bericht}

%%%%%%%%%%%%%%%%%%%%%%%%%%%%%%%%%%%%%%%%%%%%%%%%%%%%%%%%%%%%%%%%%%%%%%%%%%%%%%%
%% Angaben zur Arbeit
%%%%%%%%%%%%%%%%%%%%%%%%%%%%%%%%%%%%%%%%%%%%%%%%%%%%%%%%%%%%%%%%%%%%%%%%%%%%%%%

\newcommand{\Autor}{Simon Leitl und Mikka Jenne}
\newcommand{\MatrikelNummer}{7068806}
\newcommand{\Kursbezeichnung}{tinf17b4}

\newcommand{\FirmenName}{cjt Systemsoftware Ag}
\newcommand{\FirmenStadt}{Karlsruhe}
%\newcommand{\FirmenLogoDeckblatt}{\fbox{\includegraphics[width=3cm]{lion}}}

% Falls es kein Firmenlogo gibt:
\newcommand{\FirmenLogoDeckblatt}{}

\newcommand{\BetreuerFirma}{Titel Vorname Nachname}
\newcommand{\BetreuerDHBW}{Titel Vorname Nachname}

%%%%%%%%%%%%%%%%%%%%%%%%%%%%%%%%%%%%%%%%%%%%%%%%%%%%%%%%%%%%%%%%%%%%%%%%%%%%%%%%%%%%%

% Wird auf dem Deckblatt und in der Erklärung benutzt:
%\newcommand{\Was}{Projekt-/Studien-/Bachleorarbeit}
%\newcommand{\Was}{Projektrarbeit}
\newcommand{\Was}{Studienarbeit}
%\newcommand{\Was}{Bachleorarbeit}

%%%%%%%%%%%%%%%%%%%%%%%%%%%%%%%%%%%%%%%%%%%%%%%%%%%%%%%%%%%%%%%%%%%%%%%%%%%%%%%%%%%%%

\newcommand{\Titel}{Entwicklung einer Software zur Schaltplanerstellung in der Elektrotechnik}
\newcommand{\AbgabeDatum}{18. Mai 2020}

\newcommand{\Dauer}{12 Wochen}

% \newcommand{\Abschluss}{Bachelor of Engineering}
\newcommand{\Abschluss}{Bachelor of Science}

\newcommand{\Studiengang}{Informatik}
% \newcommand{\Studiengang}{Informatik / Angewandte Informatik}

\hypersetup{%%
  pdfauthor={\Autor},
  pdftitle={\Titel},
  pdfsubject={\Was}
}

%%%%%%%%%%%%%%%%%%%%%%%%%%%%%%%%%%%%%%%%%%%%%%%%%%%%%%%%%%%%%%%%%%%%%%%%%%%%%%%

% Wenn \includeonly{..} benutzt wird, werden nur diese Kaptitel ausgegeben.
\includeonly{
  abk
 ,kapitel1
 ,kapitel2
 ,changelog
}

%%%%%%%%%%%%%%%%%%%%%%%%%%%%%%%%%%%%%%%%%%%%%%%%%%%%%%%%%%%%%%%%%%%%%%%%%%%%%%%

% Benutzt man das "biblatex"-Paket, dann muß das hier stehen:
% siehe auch die mit BIBLATEX markierten Zeilen in bericht.sty
\bibliography{bericht}

\begin{document}

%%%%%%%%%%%%%%%%%%%%%%%%%%%%%%%%%%%%%%%%%%%%%%%%%%%%%%%%%%%%%%%%%%%%%%%%%%%%%%%%

\begin{titlepage}
\begin{center}
\vspace*{-2cm}
\FirmenLogoDeckblatt\hfill\includegraphics[width=4cm]{dhbw-logo}\\[2cm]
{\Huge \Titel}\\[1cm]
{\Huge\scshape \Was}\\[1cm]
{\large für die Prüfung zum}\\[0.5cm]
{\Large \Abschluss}\\[0.5cm]
{\large des Studienganges \Studiengang}\\[0.5cm]
{\large an der}\\[0.5cm]
{\large Dualen Hochschule Baden-Württemberg Karlsruhe}\\[0.5cm]
{\large von}\\[0.5cm]
{\large\bfseries \Autor}\\[1cm]
{\large Abgabedatum \AbgabeDatum}
\vfill
\end{center}
\begin{tabular}{l@{\hspace{2cm}}l}
Bearbeitungszeitraum	         & \Dauer 			\\
Matrikelnummer	                 & \MatrikelNummer		\\
Kurs			         & \Kursbezeichnung		\\
Ausbildungsfirma	         & \FirmenName			\\
			         & \FirmenStadt			\\
Betreuer der Ausbildungsfirma	 & \BetreuerFirma		\\
Gutachter der Studienakademie	 & \BetreuerDHBW		\\
\end{tabular}
\end{titlepage}

%%%%%%%%%%%%%%%%%%%%%%%%%%%%%%%%%%%%%%%%%%%%%%%%%%%%%%%%%%%%%%%%%%%%%%%%%%%%%%%

%%%%%%%%%%%%%%%%%%%%%%%%%%%%%%%%%%%%%%%%%%%%%%%%%%%%%%%%%%%%%%%%%%%%%%%%%%%%%%%
%% Descr:       Vorlage für Berichte der DHBW-Karlsruhe, Erklärung
%% Author:      Prof. Dr. Jürgen Vollmer, vollmer@dhbw-karlsruhe.de
%% $Id: erklaerung.tex,v 1.7 2018/10/23 08:57:55 vollmer Exp vollmer $
%% -*- coding: utf-8 -*-
%%%%%%%%%%%%%%%%%%%%%%%%%%%%%%%%%%%%%%%%%%%%%%%%%%%%%%%%%%%%%%%%%%%%%%%%%%%%%%%

% In Bachelorarbeiten muss eine schriftliche Erklärung abgegeben werden.
% Hierin bestätigen die Studierenden, dass die Bachelorarbeit, etc.
% selbständig verfasst und sämtliche Quellen und Hilfsmittel angegeben sind. Diese Erklärung
% bildet das zweite Blatt der Arbeit. Der Text dieser Erklärung muss auf einer separaten Seite
% wie unten angegeben lauten.

\newpage
\thispagestyle{empty}
\begin{framed}
\begin{center}
\Large\bfseries Erklärung
\end{center}
\medskip
\noindent
% siehe §5(3) der \enquote{Studien- und Prüfungsordnung DHBW Technik} vom 29.\,9.\,2017
Ich versichere hiermit, dass ich meine \Was mit dem Thema:
\enquote{\Titel}
selbstständig verfasst und keine anderen als die angegebenen Quellen und Hilfsmittel benutzt habe. Ich versichere zudem, dass die eingereichte elektronische Fassung mit der gedruckten Fassung übereinstimmt.
\vspace{3cm}
\noindent
\underline{\hspace{4cm}}\hfill\underline{\hspace{6cm}}\\
Ort~~~~~Datum\hfill Unterschrift\hspace{4cm}
\end{framed}

\vfill
\emph{Sofern von der Ausbildungsstätte ein Sperrvermerk gewünscht wird, ist folgende Formulierung
zu verwenden:}
\begin{framed}
\begin{center}
\Large\bfseries Sperrvermerk
\end{center}
\medskip
\noindent
Der Inhalt dieser Arbeit darf weder als Ganzes noch in Auszügen Personen
auerhalb des Prüfungsprozesses und des Evaluationsverfahrens zugänglich gemacht
werden, sofern keine anders lautende Genehmigung der Ausbildungsstätte vorliegt.
\end{framed}

%%%%%%%%%%%%%%%%%%%%%%%%%%%%%%%%%%%%%%%%%%%%%%%%%%%%%%%%%%%%%%%%%%%%%%%%%%%%%%%
\endinput
%%%%%%%%%%%%%%%%%%%%%%%%%%%%%%%%%%%%%%%%%%%%%%%%%%%%%%%%%%%%%%%%%%%%%%%%%%%%%%%


%%%%%%%%%%%%%%%%%%%%%%%%%%%%%%%%%%%%%%%%%%%%%%%%%%%%%%%%%%%%%%%%%%%%%%%%%%%%%%%

\begin{abstract}
Dieses \LaTeX-Dokument kann als Vorlage für einen Praxis- oder Projektbericht, eine Studien- oder
Bachelorarbeit dienen.

Zusammengestellt von Prof.\,Dr.\,Jürgen Vollmer \email{juergen.vollmer@dhbw-karlsruhe.de}\\
\url{https://www.karlsruhe.dhbw.de}. Die jeweils aktuellste Version dieses \LaTeX-Paketes ist immer
auf der \emph{FAQ-Seite} des Studiengangs Informatik zu finden:
\url{https://www.karlsruhe.dhbw.de/inf/studienverlauf-organisatorisches.html} $\to$ \emph{Formulare und Vorlagen}.

\centering Stand \verb+$Date: 2018/10/23 09:10:06 $+
\end{abstract}

\newpage
\tableofcontents           % Inhaltsverzeichnis hier ausgeben
\listoffigures             % Liste der Abbildungen
\listoftables              % Liste der Tabellen
\lstlistoflistings         % Liste der Listings
\listofequations           % Liste der Formeln

% Jetzt kommt der "eigentliche" Text
\include{abk}              % Abkürzungsverzeichnis
\include{kapitel1}
\include{kapitel2}

% Ab hier beginnt der Anhang
\appendix
\addcontentsline{toc}{chapter}{Anhang}

\addcontentsline{toc}{chapter}{Index}
\printindex

\addcontentsline{toc}{chapter}{Literaturverzeichnis}

% Haben Sie das "biblatex"-Paket nicht installiert, benutzen Sie folgendes:
% Ohne das "biblatex"-Paket (s. bericht.sty) produziert folgendes
% "deutsche" Zitate in Literaturverzeichnissen gemaß der Norm DIN 1505,
% Teil 2 vom Jan. 1984.
% Die Zitatmarken werden alphabetisch nach Verfassern
% sortiert und sind durch abgekürzte Verfasserbuchstaben plus
% Erscheinungsjahr in eckigen Klammern gekennzeichnet.

% \bibliographystyle{alphadin}
% \bibliography{bericht}

%%%%%%%%%%%%%%%%%%%%%%%%%%%%%%%%%%%%%%%5
% BIBLATEX
% Benutzt man das "biblatex"-Paket, muß man folgendes schreiben:
\def\refname{Literaturverzeichnis}
\printbibliography
%%%%%%%%%%%%%%%%%%%%%%%%%%%%%%%%%%%%%%%5


\include{changelog}

\newpage
\addcontentsline{toc}{chapter}{Liste der ToDo's}
\listoftodos[Liste der ToDo's]


\end{document}
